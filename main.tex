\documentclass[aspectratio=169]{beamer}

% Beamer template
\usetheme{pure}

% Title page
\title[Pure Beamer]
{A Simple beamer template: Pure}

\author[Someone, Another]
{Someone\inst{1} \and Another one\inst{2}}

\institute[Universities of Somewhere and Elsewhere]
{
	\inst{1}%
	Department of Mathematics\\
	Somewhere University
	\and
	\inst{2}%
	Department of Mathematics\\
	Elsewhere University
}

\date[Beamer 2022]
{Conference on Beamer, 2022}

\begin{document}
%-----------------------------------------------------------
\begin{frame}
	\titlepage
\end{frame}

%-----------------------------------------------------------

\begin{frame}
	\tableofcontents
	% You might wish to add the option [pausesections], if you
	% have too many pages in your slide.
\end{frame}

%-----------------------------------------------------------

\section{Motivation}

%-----------------------------------------------------------

\begin{frame}[fragile]{Motivation}{This is a subtitle}
You should place the \verb|beamerthemepure.tex| with your main 
file together. Then add
\begin{verbatim}
\usetheme{pure}
\end{verbatim}
	in your preamble.
\end{frame}

%-----------------------------------------------------------

\section{Results}

\subsection{Environments}
%-----------------------------------------------------------

\begin{frame}{theorem environments}
	\begin{theorem}[a theorem]
		A theorem here.
	\end{theorem}

	\begin{definition}[a definition]
		A definition here.
	\end{definition}

	\begin{proof}[a proof]
		A proof here.
	\end{proof}
\end{frame}

%-----------------------------------------------------------

\begin{frame}{blocks enviroments}{Including blocks, alert, examples}
	\begin{block}{block title}
		A block here.
	\end{block}

	\begin{alertblock}{alert title}
		An alert here.
	\end{alertblock}
	
	\begin{examples}[example]
		An example here.
	\end{examples}
\end{frame}

%-----------------------------------------------------------

\subsection{Lists}
%-----------------------------------------------------------
\begin{frame}[t]{different lists}
\begin{columns}
	\column{.5\textwidth}
	\begin{enumerate}
		\item item1
		\item item2
		\item item3
	\end{enumerate}
	
	\begin{itemize}
		\item item1
		\item item2
		\item item3
	\end{itemize}
	
	\pause
	
	\column{.5\textwidth}
	\begin{description}[l]
		\item[item1] This is item1.
		\item[item2] This is item2.
	\end{description}	
\end{columns}
\end{frame}

%-----------------------------------------------------------

\subsection{Mathematical equations}
%-----------------------------------------------------------

\begin{frame}{Equations}
	\begin{equation} \label{eq:equ1}
		4\pi w^{\infty}\left(-d, z\right) = u^{sc}\left(z, d\right),\ z \in \mathbb{R}^{3}\backslash \overline{D},\ d \in \mathbb{S}^{2}
	\end{equation}	
	
	\begin{itemize}
	    \item By the equation (\ref{eq:equ1}), we have that ... .
	    \item For details, we refer to \href{www.google.com}{Google}!
	\end{itemize}
\end{frame}

%-----------------------------------------------------------

\begin{frame}{Misc}
	\begin{itemize}
		\item<1-> You can use \texttt{alert}  to emph \alert{something important}.
		\item<2-> You can use \texttt{columns} to arrange your slides.
		\item<3-> Maybe use different colors: \textbf{\textcolor{purple}{purple}} for 
		          emph; \textcolor{teal}{teal} for references.
		\item<4-> You can use \texttt{columns} to arrange your slides.
		\item<5-> ...
	\end{itemize}
	
\end{frame}

%-----------------------------------------------------------

\section{Conclusions}

%-----------------------------------------------------------

\begin{frame}{Conclusions}
	\begin{center}
		Enjoy \LaTeX\ and Beamer!
	\end{center}
\end{frame}

\end{document}
